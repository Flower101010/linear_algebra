%%%%%%%%%%%%%%%%%%%%%%%%%%%%%%%%%%%%%%%%%
% fphw Assignment
% LaTeX Template
% Version 1.0 (27/04/2019)
%
% This template originates from:
% https://www.LaTeXTemplates.com
%
% Authors:
% Class by Felipe Portales-Oliva (f.portales.oliva@gmail.com) with template 
% content and modifications by Vel (vel@LaTeXTemplates.com)
%
% Template (this file) License:
% CC BY-NC-SA 3.0 (http://creativecommons.org/licenses/by-nc-sa/3.0/)
%
%%%%%%%%%%%%%%%%%%%%%%%%%%%%%%%%%%%%%%%%%

%----------------------------------------------------------------------------------------
%	PACKAGES AND OTHER DOCUMENT CONFIGURATIONS
%----------------------------------------------------------------------------------------

\documentclass[
	12pt, % Default font size, values between 10pt-12pt are allowed
	%letterpaper, % Uncomment for US letter paper size
	%spanish, % Uncomment for Spanish
]{fphw}

%----------------------------------------------------------------------------------------
% PACKAGES            																						  %
%----------------------------------------------------------------------------------------


% Template-specific packages
\usepackage[T1]{fontenc} % Output font encoding for international characters
\usepackage{amsfonts}
\usepackage{graphicx} % Required for including images
\usepackage{booktabs} % Required for better horizontal rules in tables
\usepackage{listings} % Required for insertion of code
\usepackage{enumerate} % To modify the enumerate environment
\usepackage{amsmath, amsthm, amssymb} % math stuff
\usepackage{tikz} % graph
\usetikzlibrary{decorations.markings}
\usepackage[UTF8,scheme=plain]{ctex}%中文支持
\usepackage{silence}
\WarningFilter{xeCJK}{Redefining CJKfamily `\CJKrmdefault'}
\usepackage{physics2}%更快、更简单地排版数学公式的命令
\usephysicsmodule{diagmat}
%----------------------------------------------------------------------------------------
% MY COMMANDS   
%----------------------------------------------------------------------------------------																						  

\newcommand{\Z}{\mathbb{Z}}
\newcommand{\R}{\mathbb{R}}
\newcommand{\C}{\mathbb{C}}
\newcommand{\F}{\mathbb{F}}
\newcommand{\bigO}{\mathcal{O}}
\newcommand{\Real}{\mathcal{Re}}
\newcommand{\poly}{\mathcal{P}}
\newcommand{\mat}{\mathcal{M}}
\DeclareMathOperator{\Span}{span}
\newcommand{\Hom}{\mathcal{L}}
\DeclareMathOperator{\Null}{null}
\DeclareMathOperator{\Range}{range}
\newcommand{\defeq}{\vcentcolon=}
\newcommand{\restr}[1]{|_{#1}}

%----------------------------------------------------------------------------------------
% SECTION NUMBERING																				           
%----------------------------------------------------------------------------------------

\renewcommand\thesection{\Alph{section}}

%----------------------------------------------------------------------------------------
% THEOREMS/LEMMAS/ETC.         																			  
%----------------------------------------------------------------------------------------
\newtheorem{thm}{Theorem}
\newtheorem*{thm-non}{Theorem}
\newtheorem{lemma}[thm]{Lemma}
\newtheorem{corollary}[thm]{Corollary}

%----------------------------------------------------------------------------------------
%	ASSIGNMENT INFORMATION
%----------------------------------------------------------------------------------------

%\setCJKmainfont{Source Han Sans SC}%中文字体设置--非衬线体
%\setCJKmainfont{Source Han Serif SC}%中文字体设置--衬线体


\title{Linear Maps} % Assignment title

\author{Flower} % Student name

\date{} % Due date

\institute{University of Mars \\ Institute of Intergalactic Travel} % Institute or school name

\class{Linear Algebar} % Course or class name

%\professor{Dr. Albert Einstein} % Professor or teacher in charge of the assignment

%----------------------------------------------------------------------------------------

\begin{document}

\maketitle % Output the assignment title, created automatically using the information in the custom commands above

%----------------------------------------------------------------------------------------
%	ASSIGNMENT CONTENT
%----------------------------------------------------------------------------------------

\section{The Vector Space of Linear Maps}
\section{The Vector Space of Linear Maps}

\begin{problem}{3}
  假设 $T\in \Hom(\F^n,\F^m)$.  
  证明存在$A_{j,k}\in\F$ ,其中 $j=1,\dots,m$
   $k=1,\dots,n$, 使得
  \begin{equation*}
  T(x_1,\dots,x_n) = (A_{1,1}x_1 + \dots + A_{1,n}x_n,\dots, A_{m,1}x_1 + \dots + A_{m,n}x_n)
  \end{equation*}
  对于每一个 $(x_1,\dots,x_n)\in\F^n$都成立.
\end{problem}
  %------------------------------------------------
\begin{proof}[Proof]
  对于任意的 $x\in\F^n$,我们可以写 
  \begin{equation*}
    x = x_1 e_1 + \dots + x_n e_n,
  \end{equation*}
  其中 $e_1,\dots,e_n$ 是 $\F^n$ 的标准基.  
  因为 $T$ 是线性的,我们有
  \begin{equation*}
    Tx = T(x_1 e_1 + \dots +x_n e_n) = x_1 Te_1 + \dots + x_n Te_n.
  \end{equation*}
  现在对于 $Te_k\in\F^m$, 其中 $k=1,\dots, n$, 都存在 
  $A_{1,k},\dots, A_{m,k}\in\F$ 使得
  \begin{align*}
    Te_k &= A_{1,k}e_1 + \dots + A_{m,k}e_m\\
        &= \left(A_{1,k}, \dots, A_{m,k}\right)
  \end{align*}
  因此
  \begin{equation*}
    x_kTe_k = \left(A_{1,k}x_k, \dots, A_{m,k}x_k\right).
  \end{equation*}
  所以我们有
  \begin{align*}
    Tx &= \sum_{k = 1}^n \left(A_{1,k}x_k, \dots, A_{m,k}x_k\right)\\
       &= \left(\sum_{k = 1}^nA_{1,k}x_k, \dots, \sum_{k = 1}^nA_{m,k}x_k \right),
  \end{align*}
  就证得存在$A_{j,k}\in\F$ ,其中 $j=1,\dots,m$ 并且 $k=1,\dots,n$ 使得等式成立.
\end{proof}

\begin{problem}{4}
  设 $T\in \Hom(V,W) $ 并且 $v_1,\dots,v_m$ 是$V$中的
  一组向量,其使得$Tv_1,\dots,Tv_m$在W上的线性独立。证明
 $v_1,\dots,v_m$线性独立.
\end{problem}

\begin{proof}
  假设$v_1,\dots,v_m$不线性独立,则有方程
  \begin{equation*}
    a_1v_1+\dots+a_mv_m = 0
  \end{equation*}
  有一组$a_j$不全为零的解,接下来
  \begin{displaymath}
    T(a_1v_1+\dots+a_mv_m) =a_1Tv_1+\dots+a_mTv_m =0
  \end{displaymath}
  则存在一组不全为零的$a_j$使得上式成立。与条件矛盾,故假设
  不成立。原命题正确.
\end{proof}


\begin{problem}{7}
  证明如果$\dim V = 1$ 并且 $T\in \Hom(V,V)$,
  存在$\lambda\in \F$ 使得对于任意的$v\in V$
  都有$Tv=\lambda v$.
\end{problem}

\begin{proof}
  因为$\dim V = 1$,所以$V$的基为单向量,设为$e$
  则存在$\alpha,\lambda$使得下式成立
  \begin{displaymath}
    Tv=T(\alpha e)=\alpha Te =\alpha \lambda e = \lambda v
  \end{displaymath}
  其中$\lambda$即为所需。原命题证明完毕
  
\end{proof}


\begin{problem}{8}
  找到一个$\R^2\rightarrow\R $函数$\varphi$,且对于
  任意的$a\in\R$ 和 $V\in\R^2$ 都满足
  \begin{displaymath}
    \varphi(av)=a \varphi(v)
  \end{displaymath}
  并且$\varphi$不是线性的.
\end{problem}

\begin{proof}
  找到如下函数
  \begin{displaymath}
    \varphi = \ln(xy) ,\ (x,y)\in \R^2
  \end{displaymath}
  则可有
  \begin{displaymath}
    \varphi(av) = \varphi[(ax,ay)] = a\ln(xy) = a \varphi(v)
  \end{displaymath}
  但是$\varphi$ 不满足
  \begin{displaymath}
    \varphi(\nu+\omega) \neq \varphi(\nu) +\varphi(\omega)
  \end{displaymath}
  故$\varphi$不是线性的.
  
\end{proof}


\begin{problem}{9}
  给出一个$\C\rightarrow\C$的函数$\varphi$,
  对于所有的$z,\omega\in\C$有
  \begin{displaymath}
    \varphi(\omega+z) = \varphi(\omega) +\varphi(z)
  \end{displaymath}
  但是$\varphi$不是线性的.
\end{problem}

\begin{proof}
  定义
  \begin{align*}
    \varphi:\C&\to\C\\
    x + yi &\mapsto x - yi. 
  \end{align*}
  然后对于 $x_1 + y_1i, x_2 + y_2i\in\C$, 有
  \begin{align*}
    \varphi((x_1 + y_1i) + (x_2 + y_2i)) &= \varphi((x_1 + x_2) + (y_1 + y_2)i)\\ 
                                         &= (x_1 + x_2) - (y_1 + y_2)i\\ 
                                         &= (x_1 - y_1)i + (x_2 - y_2)i\\ 
                                         &= \varphi(x_1 + y_1i) + \varphi(x_2 + y_2i)
  \end{align*}
  所以 $\varphi$ 满足加法分配律.然而
  \begin{align*}
    \varphi(i\cdot i) = \varphi(-1) = -1
  \end{align*}
  此外
  \begin{equation*}
    i\cdot\varphi(i) = i(-i) = 1 
  \end{equation*}
  则 $\varphi$ 不是线性的.
\end{proof}

\begin{problem}{10}
  设$U$ 是$V$的子集且$U\neq V$.
  设$S\in\Hom(U,W)$且$S\neq 0$.
  定义 $T:V\rightarrow W$
  \begin{equation}
    Tv=\left\{
      \begin{aligned}
        S\nu &, \ \mathrm{if}\ v\in U \\
        0 &,\ \mathrm{if}\ v\in V\ \mathrm{and}\ v\notin U
      \end{aligned}
    \right.
  \end{equation}
  证明$T$不是$V$上的线性映射.
\end{problem} 


\begin{proof}
  令
  $$v\in U,\ \omega\in V \ \mathrm{and} \ \omega \notin U $$
  所以有
  \begin{displaymath}
    v+\omega \in V\ \mathrm{and} \ v+\omega \notin U  
  \end{displaymath}
  故下面不等式成立
  \begin{displaymath}
    T(v+\omega) \neq Tv+T\omega
  \end{displaymath}
  故$T$ 不是$V$上的线性映射.
\end{proof}

\begin{problem}{11}
  Suppose $V$ is  finite-dimensional. Prove that every linear
  map on a subspace of $V$ can be extended to a linear map
  on $V$. In other words, show that if U is a suspace of $V$ and
  $S\in\Hom(U,W)$, then there exists $T\in\Hom(V,W)$ such
  that $Tu=Su$ for all $u\in U$.
\end{problem}

\begin{proof}
  设$U$为$V$的子集,则$U$存在一组基向量
  $v_1,\dots,v_m$. 将该组基向量扩展为
  $v+1,\dots,v_m,v_{m+1},\dots,v_n$,并且为$V$的一组基向量.
  则易知对于任意的$z\in V$,都可以找到一组
  $a_1,\dots,a_n \in \F$ 使得 $z=\sum_{k=1}^{n}a_k v_k$.

  我们设
  \begin{align*}
    T:V&\rightarrow w\\
    \displaystyle\sum_{k=1}^{n}a_k v_k &\rightarrow \displaystyle\sum_{k=1}^{m}a_k S v_k + \displaystyle\sum_{k=m+1}^{n} a_k v_k
  \end{align*}
  显然$Tu=Su$,仅需证明$T$ 是线性映射即可.

  证明满足分配律
  设
  \[
    z_1 = a_1v_1+\dots+a_nv_n,\ z_2 = b_1v_1+\dots+b_nv_n
  \]

  则
  \begin{align*}
    T(z_1+z_2) &= \displaystyle\sum_{k=1}^{m} (a_k+b_k)Sv_k + \displaystyle\sum_{k=m+1}^{n} (a_k+b_k) v_k\\
               &= \displaystyle\sum_{k=1}^{m} a_kSv_k +\displaystyle\sum_{k=m+1}^{n} a_k v_k + \displaystyle\sum_{k=1}^{m} b_kSv_k +\displaystyle\sum_{k=m+1}^{n} b_k v_k\\
               &=Tz_1+Tz_2
  \end{align*}

  证明满足数量乘法,设$\lambda\in \F$,则
  \begin{align*}
    T(\lambda z) &= \displaystyle\sum_{k=1}^{m} \lambda a_kSv_k + \displaystyle\sum_{k=m+1}^{n} \lambda a_k v_k\\
                   &= \lambda\displaystyle\sum_{k=1}^{m}  a_kSv_k + \lambda\displaystyle\sum_{k=m+1}^{n}  a_k v_k\\
                   &= \lambda Tz
  \end{align*}

  \end{proof}


\begin{problem}{12}
  设$V$是有限维的向量空间且$\dim V > 0 $, 并设$W$
  是无限维的向量空间。证明$\Hom(V,W)$是无限维的向
  量空间.
\end{problem}

\begin{proof}
  设$v\in V $ 并且设$\omega_1,\dots$ 是$W$的一组
  基.则对于任意的$m$,$\omega_1,\dots,\omega_m$
  都独立.(见2a/14)
  
  定义$T_j(v)=\omega_j$,显然$T_j\in\Hom(V,W)$仅需证明
  数列$T_1,\dots$ 中,任意的$m$,$T_1,\dots,T_m$
  都独立. 设
  
  \[
    a_1T_1+\dots+a_nT_n = 0
  \]
  仅需说明$a_j$全为零是唯一解.
  \[
    a_1T_1v+\dots+a_nT_nv = a_1\omega_1+\dots+a_n\omega_n =0
  \]
  因为对于任意的$m$,$\omega_1,\dots,\omega_m$
  都独立.说明$a_j$全为零是唯一解.
  证得$\Hom(V,W)$是无限维的向
  量空间.
\end{proof}


\begin{problem}{13}
  设$v_1,\dots, v_n$ 是$V$ 上的一组线性独立的数列.
  同时设$W\neq{0}$. 证明存在$\omega_1,\dots,\omega_m \in W $
  使得不存在$T\in\Hom(V,W)$ 满足 $Tv_k=\omega_k,\ (k=1,\dots,m)$. 
\end{problem}

\begin{proof}
  假设对于所有的$\omega_1,\dots,\omega_m \in W $
  使得存在$T\in\Hom(V,W)$ 满足 
  $Tv_k=\omega_k,\ (k=1,\dots,m)$.
  上面假设说明$\omega_k$独立。
  显然可以找到一组
  $\omega_1,\dots,\omega_m \in W $,
  且它们不相互独立。
  故假设不成立,所以原命题正确.
\end{proof}

\begin{problem}{14}
  设$V$是有限维的向量空间且$\dim V > 2 $, 
  证明存在$S,T\in\Hom(V,V)$ 满足 $ST\neq TS$.
\end{problem}

\begin{proof}
  显然只要找到两个矩阵$A,B$,使得$AB\neq BA$即可.
\end{proof}

\section{Null Space and Ranges}
\begin{problem}{2}
假设$V$是一个线性空间并且$S,T\in\Hom(V,V)$满足
\[
	\Range S \subset \Null T .
\]
证明$(ST)^2=0$
\end{problem}

\begin{proof}
	因为$\Range S \subset \Null T$,所以
	对于任意的$v\in V$都有$TSv=0$.
	所以对于任意的$u\in V$都有
	\[
		(ST)^2u=S[(TS)Tu]=S0=0.
	\]
	证得$(ST)^2=0$
\end{proof}

\begin{problem}{13}
设$T$ 是一个$\F^4$到$\F^2$的线性映射,且满足
\begin{displaymath}
	\Null T = {(x_1,x_2,x_3,x_4)\in\F^4:x_1=5x_2\ \mathrm{and}\ x_3=7x_4}
\end{displaymath}
证明T是满射的.
\end{problem}

\begin{proof}
	显然$\dim \Null T = 2$,故
	\begin{displaymath}
		\dim \Range T =2.
	\end{displaymath}
	我们易证下面引理
	\
	\begin{lemma}
		设$U$是$V$的子空间,若$\dim U =\dim V $,
		则$U=V$.
	\end{lemma}
	\

	故有
	\begin{displaymath}
		\Range T = \F^2
	\end{displaymath}
	说明是满射的.
\end{proof}

\begin{problem}{20}
设$W$是有限维的并且$T\in\Hom(V,W)$.
证明$T$是单射的当且仅当存在$S\in\Hom(W,V)$
使得$ST$为$V$上的恒等映射.
\end{problem}

\begin{proof}
	定义
	\begin{displaymath}
		S = \begin{cases}
			v, & \text{ if } u\in Tv                        \\
			0, & \text{ if } u\in W \text{ and } u\notin Tv
		\end{cases}
	\end{displaymath}
	这样就容易证明原命题成立
\end{proof}

\begin{problem}{21}
设$W$是有限维的并且$T\in\Hom(V,W)$.
证明$T$是满射的当且仅当存在$S\in\Hom(W,V)$
使得$TS$为$W$上的恒等映射.
\end{problem}

\begin{proof}
	$(\Rightarrow)$  Suppose $T\in\Hom(V,W)$ is surjective, so that $W$ is necessarily finite-dimensional as well.  Let $v_1,\dots, v_m$ be a basis of $V$ and let $n=\dim W$, where $m\geq n $ by surjectivity of $T$.  Note that
	\begin{equation*}
		Tv_1,\dots, Tv_m
	\end{equation*}
	span $W$.  Thus we may reduce this list to a basis by removing some elements (possibly none, if $n = m$).  Suppose this reduced list were $Tv_{i_1},\dots, Tv_{i_n}$ for some $i_1,\dots, i_n\in\{1,\dots, m\}$.  We define $S\in\Hom(W,V)$ by its behavior on this basis
	\begin{equation*}
		S(Tv_{i_k}) := v_{i_k} \text{ for }k = 1,\dots, n.
	\end{equation*}
	Suppose $w \in W$.  Then there exist $a_1,\dots, a_n\in\F$ such that
	\begin{equation*}
		w = a_1 Tv_{i_1} + \dots + a_nTv_{i_n}
	\end{equation*}
	and thus
	\begin{align*}
		TS(w) & = TS\left(a_1 Tv_{i_1} + \dots + a_nTv_{i_n}\right)               \\
		      & = T\left(S\left( a_1 Tv_{i_1} + \dots + a_nTv_{i_n}\right)\right) \\
		      & = T\left( a_1 S(Tv_{i_1}) + \dots + a_nS(Tv_{i_n})\right)         \\
		      & = T(a_1 v_{i_1} + \dots + a_nv_{i_n})                             \\
		      & = a_1 Tv_{i_1} + \dots + a_nTv_{i_n}                              \\
		      & = w,
	\end{align*}
	and so $TS$ is the identity map on $W$.
	\par $(\Leftarrow)$ Suppose there exists $S\in\Hom(W,V)$ such that $TS\in\Hom(W,W)$ is the identity map, and suppose by way of contradiction that $T$ is not surjective, so that $\dim \Range TS < \dim W$.  By the Fundamental Theorem of Linear Maps, this implies
	\begin{align*}
		\dim W & = \dim\Null TS + \dim \Range TS \\
		       & < \dim\Null TS + \dim W
	\end{align*}
	and hence $\dim\Null TS > 0$, a contradiction, since the identity map can only have trivial null space.  Thus $T$ is surjective, as desired.
\end{proof}

\section{Matrices}
%----------------------------------------------------------------------------------------
%	Problem 1
%----------------------------------------------------------------------------------------
\begin{problem}{1}
设$V$和$W$都是有限维的且$T\in\Hom(V,W)$.
证明对于$ V $和$ W $的任意基 ,
$T$ 的矩阵
都至少有 $\dim\Range T $个非零元 .
\end{problem}
\begin{proof}
	设$v_1,\dots,V_n$为$V$的基,
	$w_1,\dots,w_m$为$W$的基,$r=\dim \Range T$ 和
	$s = \dim \Null T$.所以$V$的基中有$s$个映射到$0$,
	$r$个映射为非零.
	若$Tv_k\neq 0 $,则存在唯一一组不不全为零(最少有一个不为零)的$A_{j,k}\in \F$使得
	\[
		Tv_k = \displaystyle\sum_{j=1}^{m}A_{j,k}w_j
	\]
	能满足$Tv_k\neq 0 $的基向量有$r=\dim \Range T$个.
	故最少有 $\dim\Range T $个非零元.

\end{proof}
%----------------------------------------------------------------------------------------
%	Problem 3
%----------------------------------------------------------------------------------------

\begin{problem}{3}
设$V$和$W$都是有限维的且$T\in\Hom(V,W)$.
证明存在一个$V$的基和$W$的基,使得关于这
些基,$\mat(T)$除了第$j$行第$j$列($1\leq j \leq \dim \Range T$)的元素
为1,其余均为0.
\end{problem}
\begin{proof}
	设 $R$ 是$V$ 的子空间且满足
	\begin{equation*}
		V = R\oplus \Null T,
	\end{equation*}
	设 $r_1,\dots, r_m$ 为$R$的基 (其中$m =\dim\Range T$),
	并设 $v_1,\dots, v_n$为 $\Null T$的基  (其中 $n = \dim\Null T$).
	那么 $r_1,\dots, r_m, v_1,\dots,v_n$为$V$的基.
	而且也易得 $Tr_1,\dots, Tr_m$ 是 $\Range T$的基.
	因此 扩展上述的基使之成为$W$的基.
	设 $Tr_1,\dots, Tr_m, w_1,\dots, w_p$ 是这样的基(其中 $p = \dim W - m$).
	那么对于 $j = 1,\dots m$, 我们有
	\begin{equation*}
		Tr_j = \left(\sum_{i = 1}^m\delta_{i,j}\cdot Tr_t\right) + \left(\sum_{k = 1}^p 0\cdot w_k\right),
	\end{equation*}
	其中$\delta_{i,j}$ 是克罗内克函数.
	因此在$\mat(T)$的第$j$行中只有第$j$列为$0$,
	其中 $j$取 从 $1$ 到 $m = \dim\Range T$任意值.
	因此 $Tv_1 = \dots = Tv_n = 0$,
	$\mat(T)$的剩余行全为零 .
	因此 $\mat(T)$有所需的形式 .
\end{proof}


\section{可逆性与同构}
%----------------------------------------------------------------------------------------
%	Problem 1
%----------------------------------------------------------------------------------------


\begin{problem}{1}
Suppose $T\in\Hom\left( U,V \right) $ and
$S\in\Hom\left( V,W \right) $ are both invertible
liner maps. Prove that $ST\in\Hom\left( U,W \right) $
is invertible and $\left( ST \right)^{-1}=T^{-1}S^{-1}$.
\end{problem}


\begin{proof}
	易知\[
		SS^{-1}=I \quad TT^{-1}=I
	\]
	仅需说明$ST\cdot T^{-1}S^{-1}=I$即可.
	显然
	\[
		ST\cdot T^{-1}S^{-1}=SIS^{-1}=SS^{-1}=I
	\]
\end{proof}

%----------------------------------------------------------------------------------------
%	Problem 2
%----------------------------------------------------------------------------------------
\begin{problem}{2}
Suppose $V$ is finite-dimensional and $\dim V > 1$.
Prove that the set of noninvertible operators on $V$
is not a subspace of $\Hom(V)$.
\end{problem}

\begin{proof}
	\[
		\pdiagmat [ empty = {} ]
		{ 1, 1, 0, 0 } +\pdiagmat [ empty = {} ]{ 0, 0, 1, 1 } =\pdiagmat [ empty = {} ]{ 1, 1, 1, 1 }
	\]

	上面的例子说明两个不可逆的映射的和为可逆映射,故不可逆
	算子构成的集合不是$\Hom(V)$的子空间.
\end{proof}
%----------------------------------------------------------------------------------------
%	Problem 3
%----------------------------------------------------------------------------------------
\begin{problem}{3}
Suppose $V$ is finite-dimensional, $U$ is a subspace
of $V$, and $S\in\Hom(U,V)$. Prove there exists an
invertible operator $T\in\Hom(V)$ such that $Tu = Su$
for every $u\in U$ if and only if $S$ is injective.
\end{problem}

\begin{proof}
	先假设$Tu = Su$,则有
	\[
		T^{-1}Tu =u =T^{-1}Su
	\]
	易知$T^{-1}$是单射的.
	假设存在$u_1,u_2$使得$Su_1=Su_2$
	故:\[
		T^{-1}Su_1 =T^{-1}Su_2.
	\]
	可得
	\[
		u_1 = u_2.
	\]
	假设不成立. 所以 $S$ 是单射的.

	假设$S$ 是单射的. 设$RSu=u$.
	接下来仅需$R$是$V$上的可逆算子即可.
	显然$R\in\Hom(V,V)$,仅需说明单射性.因为
	\[
		RSu=u
	\]
	所以$R$在$\Span S $ 上是单射的。
	我们仅需定义一下$R$就可以使得其在$V$的其他部分单射。
	因此可以说明$T=R^{-1}$.
\end{proof}



\end{document}
