%%%%%%%%%%%%%%%%%%%%%%%%%%%%%%%%%%%%%%%%%
% fphw Assignment
% LaTeX Template
% Version 1.0 (27/04/2019)
%
% This template originates from:
% https://www.LaTeXTemplates.com
%
% Authors:
% Class by Felipe Portales-Oliva (f.portales.oliva@gmail.com) with template 
% content and modifications by Vel (vel@LaTeXTemplates.com)
%
% Template (this file) License:
% CC BY-NC-SA 3.0 (http://creativecommons.org/licenses/by-nc-sa/3.0/)
%
%%%%%%%%%%%%%%%%%%%%%%%%%%%%%%%%%%%%%%%%%

%----------------------------------------------------------------------------------------
%	PACKAGES AND OTHER DOCUMENT CONFIGURATIONS
%----------------------------------------------------------------------------------------

\documentclass[
	12pt, % Default font size, values between 10pt-12pt are allowed
	%letterpaper, % Uncomment for US letter paper size
	%spanish, % Uncomment for Spanish
]{fphw}

%----------------------------------------------------------------------------------------
% PACKAGES            																						  %
%----------------------------------------------------------------------------------------


% Template-specific packages
\usepackage[T1]{fontenc} % Output font encoding for international characters
\usepackage{mathpazo} % Use the Palatino font
\usepackage{graphicx} % Required for including images
\usepackage{booktabs} % Required for better horizontal rules in tables
\usepackage{listings} % Required for insertion of code
\usepackage{enumerate} % To modify the enumerate environment
\usepackage{amsmath, amsthm, amssymb} % math stuff
\usepackage{tikz} % graph
\usetikzlibrary{decorations.markings}
\usepackage[UTF8,fontset=none,scheme=plain]{ctex}%中文支持
\usepackage{silence}
\WarningFilter{xeCJK}{Redefining CJKfamily `\CJKrmdefault'}

%----------------------------------------------------------------------------------------
% MY COMMANDS   
%----------------------------------------------------------------------------------------																						  

\newcommand{\Z}{\mathbb{Z}}
\newcommand{\R}{\mathbb{R}}
\newcommand{\C}{\mathbb{C}}
\newcommand{\F}{\mathbb{F}}
\newcommand{\bigO}{\mathcal{O}}
\newcommand{\Real}{\mathcal{Re}}
\newcommand{\poly}{\mathcal{P}}
\newcommand{\mat}{\mathcal{M}}
\DeclareMathOperator{\Span}{span}
\newcommand{\Hom}{\mathcal{L}}
\DeclareMathOperator{\Null}{null}
\DeclareMathOperator{\Range}{range}
\newcommand{\defeq}{\vcentcolon=}
\newcommand{\restr}[1]{|_{#1}}

%----------------------------------------------------------------------------------------
% SECTION NUMBERING																				           
%----------------------------------------------------------------------------------------

\renewcommand\thesection{\Alph{section}}

%----------------------------------------------------------------------------------------
% THEOREMS/LEMMAS/ETC.         																			  
%----------------------------------------------------------------------------------------
\newtheorem{thm}{Theorem}
\newtheorem*{thm-non}{Theorem}
\newtheorem{lemma}[thm]{Lemma}
\newtheorem{corollary}[thm]{Corollary}

%----------------------------------------------------------------------------------------
%	ASSIGNMENT INFORMATION
%----------------------------------------------------------------------------------------

%\setCJKmainfont{Source Han Sans SC}%中文字体设置--非衬线体
\setCJKmainfont{Source Han Serif SC}%中文字体设置--衬线体


\title{Linear Maps} % Assignment title

\author{Flower} % Student name

\date{} % Due date

\institute{University of Mars \\ Institute of Intergalactic Travel} % Institute or school name

\class{Linear Algebar} % Course or class name

%\professor{Dr. Albert Einstein} % Professor or teacher in charge of the assignment

%----------------------------------------------------------------------------------------

\begin{document}

\maketitle % Output the assignment title, created automatically using the information in the custom commands above

%----------------------------------------------------------------------------------------
%	ASSIGNMENT CONTENT
%----------------------------------------------------------------------------------------

\section{The Vector Space of Linear Maps}
\section{The Vector Space of Linear Maps}

\begin{problem}{3}
  假设 $T\in \Hom(\F^n,\F^m)$.  
  证明存在$A_{j,k}\in\F$ ,其中 $j=1,\dots,m$
   $k=1,\dots,n$, 使得
  \begin{equation*}
  T(x_1,\dots,x_n) = (A_{1,1}x_1 + \dots + A_{1,n}x_n,\dots, A_{m,1}x_1 + \dots + A_{m,n}x_n)
  \end{equation*}
  对于每一个 $(x_1,\dots,x_n)\in\F^n$都成立.
\end{problem}
  %------------------------------------------------
\begin{proof}[Proof]
  对于任意的 $x\in\F^n$,我们可以写 
  \begin{equation*}
    x = x_1 e_1 + \dots + x_n e_n,
  \end{equation*}
  其中 $e_1,\dots,e_n$ 是 $\F^n$ 的标准基.  
  因为 $T$ 是线性的,我们有
  \begin{equation*}
    Tx = T(x_1 e_1 + \dots +x_n e_n) = x_1 Te_1 + \dots + x_n Te_n.
  \end{equation*}
  现在对于 $Te_k\in\F^m$, 其中 $k=1,\dots, n$, 都存在 
  $A_{1,k},\dots, A_{m,k}\in\F$ 使得
  \begin{align*}
    Te_k &= A_{1,k}e_1 + \dots + A_{m,k}e_m\\
        &= \left(A_{1,k}, \dots, A_{m,k}\right)
  \end{align*}
  因此
  \begin{equation*}
    x_kTe_k = \left(A_{1,k}x_k, \dots, A_{m,k}x_k\right).
  \end{equation*}
  所以我们有
  \begin{align*}
    Tx &= \sum_{k = 1}^n \left(A_{1,k}x_k, \dots, A_{m,k}x_k\right)\\
       &= \left(\sum_{k = 1}^nA_{1,k}x_k, \dots, \sum_{k = 1}^nA_{m,k}x_k \right),
  \end{align*}
  就证得存在$A_{j,k}\in\F$ ,其中 $j=1,\dots,m$ 并且 $k=1,\dots,n$ 使得等式成立.
\end{proof}

\begin{problem}{4}
  设 $T\in \Hom(V,W) $ 并且 $v_1,\dots,v_m$ 是$V$中的
  一组向量,其使得$Tv_1,\dots,Tv_m$在W上的线性独立。证明
 $v_1,\dots,v_m$线性独立.
\end{problem}

\begin{proof}
  假设$v_1,\dots,v_m$不线性独立,则有方程
  \begin{equation*}
    a_1v_1+\dots+a_mv_m = 0
  \end{equation*}
  有一组$a_j$不全为零的解,接下来
  \begin{displaymath}
    T(a_1v_1+\dots+a_mv_m) =a_1Tv_1+\dots+a_mTv_m =0
  \end{displaymath}
  则存在一组不全为零的$a_j$使得上式成立。与条件矛盾,故假设
  不成立。原命题正确.
\end{proof}


\begin{problem}{7}
  证明如果$\dim V = 1$ 并且 $T\in \Hom(V,V)$,
  存在$\lambda\in \F$ 使得对于任意的$v\in V$
  都有$Tv=\lambda v$.
\end{problem}

\begin{proof}
  因为$\dim V = 1$,所以$V$的基为单向量,设为$e$
  则存在$\alpha,\lambda$使得下式成立
  \begin{displaymath}
    Tv=T(\alpha e)=\alpha Te =\alpha \lambda e = \lambda v
  \end{displaymath}
  其中$\lambda$即为所需。原命题证明完毕
  
\end{proof}





\end{document}
